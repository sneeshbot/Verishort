\documentclass[letterpaper,12pt]{article}
\usepackage[letterpaper,margin=1in]{geometry}
%\setlength{\topmargin}{-.5in}
%\setlength{\textheight}{9in}
%\setlength{\oddsidemargin}{.125in}
%\setlength{\textwidth}{6.25in}

\begin{document}
\title{Verishort HDL}
\author{Anish Bramhandkar \ Elba Garza \ Scott Rogowski \ Ruijie Song}
\renewcommand{\today}{December 22, 2010}
\maketitle

\newpage

\tableofcontents

\newpage

\section{An Introduction to Verishort}
    \subsection{Background}
    \indent Verilog is a very popular hardware description language (HDL) which is widely utilized by
    the electronics hardware design industry. First invented and used in the early 80s at
    Automated Integrated design Systems, Verilog was put into the public domain and
    standardized by the IEEE in 1995. This initial public version of Verilog became known as 
    Verilog-95. The language was later expanded in 2001 and 2005 to address deficiencies
    and add features resulting in Verilog-2001 and Verilog-2005, the most recent version. (A
    combined hardware description / verification language known as SystemVerilog was 
    extended from the 2005 standard but goes beyond the scope of this manual.) \\
    \indent Despite its popularity, Verilog is infamous for its repetitiveness, strange grammar, and 
    ease of bug insertion. Part of this is a factor of the nature of low-level hardware design. 
    There is a difference between languages meant to be run using gates and latches rather 
    than processors and memory. However, we believe that another part of this simply poor 
    language design and can be improved. \\
    \indent VeriShort HDL is meant to simplify the Verilog-2005 language to make it easier to read 
    and write. First, we have reduced repetitiveness in accordance with the DRY (Don�t 
    repeat yourself) philosophy by simplifying module input/output syntax and instantiation. 
    Next, we introduced some C-language features such as brackets and array-like bus 
    descriptions. We substantially simplified synchronous logic by doing away with �always� 
    syntax and replacing it with simple �if� statements. The list of reserved keywords has been 
    substantially shortened in order to make VeriShort completely synthesizable and to 
    remove rarely used features. Finally, we added a standard library of commonly used 
    electronic components like latches, multiplexers, and decoders to further reduce the 
    Verilog tedium. \\
    \indent Because of the wide adoption of Verilog and the existence of many verifiers and 
    hardware synthesizers specific to the IEEE standards, the initial goal of VeriShort will not 
    be to exist as a self contained HDL but rather to translate into clean synthesizable Verilog 
    code. In support of these efforts, a translator has been started and is expected to be 
    running by the date of December 22nd 2010.

    \subsection{Related Work}
        \subsection{Goals of Verishort}
        \subsubsection{Short}
        \subsubsection{Logical}
        \subsubsection{Clean}


\section{Tutorial}
    \subsection{A First Example}
    \subsection{Compiling and Running a Verishort file}
    \subsection{More Examples}

\section{Reference Manual}


\section{Project Plan}
    \subsection{Team Responsibilities}
    \subsection{Project Timeline}
    \subsection{Software Development Environment}
    \subsection{Project Log}
    
\section{Architectural Design}
    \subsection{Architecture}
    \subsection{the Runtime Environment}
    \subsection{Error Recovery}

\section{Testing Plan}
    \subsection{Goals}
    \subsection{Hypothesis}
    \subsection{Methods}
        \subsubsection{Phase I}
        \subsubsection{Phase II}
        \subsubsection{Phase III}
    \subsection{Tools}
    \subsection{Implementation}
        \subsubsection{Phase I}
        \subsubsection{Phase II}
        \subsubsection{Phase III}
        
\end{document}
