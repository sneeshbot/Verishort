\documentclass[letterpaper,11pt]{article}
\usepackage[letterpaper,margin=1in]{geometry}
\usepackage{array}
\usepackage{graphicx}
\newcommand{\tab}{\hspace*{2em}}
%\setlength{\topmargin}{-.5in}
%\setlength{\textheight}{9in}
%\setlength{\oddsidemargin}{.125in}
%\setlength{\textwidth}{6.25in}

\begin{document}
\title{Verishort HDL}
\author{Anish Bramhandkar \ Elba Garza \ Scott Rogowski \ Ruijie Song\\\\
\textbf{Motto}: "Whatever."}
\renewcommand{\today}{December 22, 2010}
\maketitle

\newpage

\tableofcontents

\newpage

\section{An Introduction to Verishort}
    \subsection{Background}
    
    \indent\indent Verilog is a very popular hardware description language (HDL) which is widely utilized by
    the electronics hardware design industry. First invented and used in the early 80s at
    Automated Integrated design Systems, Verilog was put into the public domain and
    standardized by the IEEE in 1995. This initial public version of Verilog became known as 
    Verilog-95. The language was later expanded in 2001 and 2005 to address deficiencies
    and add features resulting in Verilog-2001 and Verilog-2005, the most recent version. (A
    combined hardware description\slash verification language known as SystemVerilog was 
    extended from the 2005 standard but goes beyond the scope of this manual.) \\
    \indent Despite its popularity, Verilog is infamous for its repetitiveness, strange grammar, and 
    ease of bug insertion. Part of this is a factor of the nature of low-level hardware design. 
    There is a difference between languages meant to be run using gates and latches rather 
    than processors and memory. However, we believe that another part of this simply poor 
    language design and can be improved. \\
    \indent VeriShort HDL is meant to simplify the Verilog-2005 language to make it easier to read 
    and write. First, we have reduced repetitiveness in accordance with the DRY (Don't 
    repeat yourself) philosophy by simplifying module input/output syntax and instantiation. 
    Next, we introduced some C-language features such as brackets and array-like bus
    descriptions. We substantially simplified synchronous logic by doing away with \texttt{always} 
    syntax and replacing it with simple \texttt{if} statements. The list of reserved keywords has been 
    substantially shortened in order to make VeriShort completely synthesizable and to 
    remove rarely used features. Finally, we added a standard library of commonly used 
    electronic components like latches, multiplexers, and decoders to further reduce the 
    Verilog tedium. \\
    \indent Because of the wide adoption of Verilog and the existence of many verifiers and 
    hardware synthesizers specific to the IEEE standards, the initial goal of VeriShort will not 
    be to exist as a self contained HDL but rather to translate into clean synthesizable Verilog 
    code. In support of these efforts, a translator has been started and is expected to be 
    running by the date of December 22nd 2010.

    \subsection{Related Work}
        \subsection{Goals of Verishort}
        \subsubsection{Short}
        We want Verishort code to be comparibly shorter than the Verilog code it creates. We want to
        take away the humdrum of writing Verilog code and make it easier to write what we wish to write. 
        \subsubsection{Logical}
        Verishort shouldn't skimp on the general power of Verilog either. Therefore, we support the most
        common operations and help make common structures easy to write. 
        \subsubsection{Clean}
        We hope to be able to write clean and understandable code that can be, at a glance, intuitively 
        translated to its equal Verilog code. The syntax should be friendly, logical, and quick to learn.


\section{Tutorial}
    \subsection{A First Example}
    \subsection{Compiling and Running a Verishort file}
    \subsection{More Examples}

\section{Reference Manual}
    \subsection{Lexical Conventions}
        \subsubsection{Tokens}
        There are 5 classes of tokens: identifiers, keywords, numbers, operators, and other separators.
        
        Blanks, tabs, and newlines (collectively, whitespace) are ignored, except when they serve
        to separate tokens.
        
        \subsubsection{Comments}
        The characters \texttt{/*} introduce a comment, which is terminated by the characters \texttt{*/}.
        
        The characters \texttt{//} also introduce a comment, which is terminated by the newline character. Comments do not 
        nest. Lines marked as comments are discarded by the compiler.
        
        \subsubsection{Data Types}
        The primary data type is the bit, which may store the value 0 or 1. A group of bits comprises a bus. All 
        multibit binary values are treated as two's complement numbers.
        
        In for-loops (see the corresponding section), the loop variable is assumed to be a simple integer (i.e., 
        natural number).
        
        \subsubsection{Identifiers}
        An identifier is a sequence of characters that represent a wire, bus, register, parameter, or module. 
        
        An identifier may only include alphanumerical characters or the underscore character (\_). The first character 
        of an identifier may not be a number.
        
        \subsubsection{Keywords}
        The following identifiers are reserved as keywords and may not be used for any other purpose: 
        In this manual, keywords are bolded.
        \begin{itemize}
        \item{\texttt{case}}
        \item{\texttt{clock}} 
        \item{\texttt{else}}
        \item{\texttt{for}}
        \item{\texttt{if}}
        \item{\texttt{input}}
        \item{\texttt{module}}
        \item{\texttt{negedge}}
        \item{\texttt{output}}
        \item{\texttt{parameter}}
        \item{\texttt{posedge}}
        \item{\texttt{register}}
        \item{\texttt{return}}
        \item{\texttt{reset}}
        \item{\texttt{wire}}
        \end{itemize}
        \subsubsection{Numbers}
        Numbers can be either binary or integer values and are specified as follows: 
        \begin{itemize}
        \item{A sequence of digits, followed by a radix suffix (b for binary, or d for integer)}
        \item{If there is only a single 0 or 1, no suffix needs to be provided because 0 and 1 are
        equivalent in integer and binary} 
        \item{The characters 0 and 1 are valid binary digits.} 
        \item{The characters 0-9 are valid integer digits.}
        \item{Extended binary numbers are like normal binary numbers, but may also use the
        character \texttt{x} as a binary digit. They may only be used in case structures.}
        \end{itemize}
        \subsubsection{Operators}
        An operator is a token that specifies an operation on at least one operand. The operand 
        may be an expression or a constant. \\\\
        Bitwise operators:
        \begin{itemize}
        \item{\texttt{!}} 
        \item{\texttt{\& }}
        \item{\texttt{!\& }} 
        \item{\texttt{$\mid$}}
        \item{\texttt{!$\mid$}} 
        \item{\texttt{\^}}
        \item{\texttt{!\^}} 
        \item{\texttt{<<}} 
        \item{\texttt{>>}}
        \end{itemize}
        Comparison operators: 
        \begin{itemize}
        \item{\texttt{==}}
        \item{\texttt{>= }}
        \item{\texttt{<= }}
        \item{\texttt{> }}
        \item{\texttt{<}}
        \end{itemize}
        Arithmetic operators: 
        \begin{itemize}
        \item{\texttt{+}}
        \item{\texttt{- }}
        \item{\texttt{* }}
        \item{\texttt{/ }}
        \item{\texttt{\%}} 
        \end{itemize}
        Assignment operator: 
        \begin{itemize}
        \item{\texttt{=}}
        \end{itemize}
        Sign extension operator: 
        \begin{itemize}
        \item{\texttt{'}}
        \end{itemize}
        The following operators are only valid within a for-loop: 
        \begin{itemize}
        \item{\texttt{++}}
        \item{\texttt{--}}
        \end{itemize}
        \subsubsection{Buses}
        A bus represents a multibit wire. The number of bits in a bus must be determinable at compile 
        time. Buses are declared using the syntax \texttt{data\_type bus\_name[number\_of\_bits];} Where 
        data\_type is either \texttt{wire, register} or assumed to be input or output by its position in a module 
        declaration. The number of bits must be a constant. From here, any bit in the bus may be referred 
        to using the subscript syntax: \texttt{bus\_name[bit\_index],} where \texttt{bit\_index} is a constant or expression 
        that yields an integer value less than or equal to the size of the bus. 
        
        A range of bits in a bus is represented by using the index of the most significant bit in the 
        range, followed by the colon character (:), followed by the index of the least significant 
        bit in the range, as the subscript. E.g. wires 4-8 would be referred to by \texttt{bus\_name[7:3]}. 
        Reversing this order is invalid.
        
    \subsection{Assignment}
    All assignments in Verishort bind wires and ports to other wires, binary values, or decimal values. 
    These can be done en masse, such as in buses (multi-bit wires) or one by one.
    
        \subsubsection{Assigning Wires}
        
        The most basic assignment is of a single wire to a single bit value: \\\\
        \texttt{wire w1 = 0;}  \\
        \texttt{wire w2 = !w1;     // w2 == 1}\\\\
        A bundle of wires can be assigned to a multi-bit value, as long as the number of bits matches 
        the number of wires in the bundle: \\\\
        \texttt{wire w3[5] = 01010b;	 // assigning a binary value} \\
        \texttt{wire w4[4] = 10d;       // assigning a decimal value} \\
        \texttt{wire w5[10] = \{1,8\{0\},1\};	// 1000000001b using concatenation} \\\\
        This does not work and will result in an error because the left hand side and the right hand 
        side are not the same size:\\
        \texttt{wire w6[10] = 10b;} \\\\
        Note that the number of bits must always be specified for more than one bit. \\\\
        Subsets of buses can be assigned: \\\\
        \texttt{wire w6[5];} \\
        \texttt{w6[3:0] = 4d;} \\ 
        \texttt{w6[4] = 1b; // w6 == 10100b} \\\\
        The two sides of an assignment must have the same number of bits. \\\\
        The assignment operator returns the right operand and associates from 
        right to left, making the following possible:  \\\\
        \texttt{wire w7[5], w8[3];} \\
        \texttt{w7[2:0] = w[8] = 010b;} \\
    
        \subsubsection{Binding Ports}
        Ports are bound to wires when instantiating a module by setting the module’s parameter name 
        equal to the wire or value to which it should be bound. Like assigning wires, these bindings 
        can be in whole or in part. The value of a port, or part of a port, that has not been bound 
        is assumed to be 0. \\\\
        \texttt{module m1(input in1[5], in2; output out[5]) \{ ... \} // declared somewhere} \\
        \texttt{\slash\slash now, inside of a calling module} \\
        \texttt{wire w7[5];} \\
        \texttt{m1(in1[3:0] = w6[4:1], in2 = 1b; out = w7);}
    
    \subsection{Expressions \& Operators}
    The physical logic of VeriShort is described using expressions which are made up of one 
    or more operators and operands. An operand can be either a single bit or a bus. 
    All expressions will return a bit or bus that is then be assigned to a wire or output 
    (see Assignment) or returned in an output. This section will detail operators ordered from the 
    most basic building blocks to complex operations.
    
        \subsubsection{Concatenation, Replication \& Splitting}
        To place two or more bits or buses together into a single bus, the concatenation syntax is used. \\
        \texttt{wire a = 0;} \\
        \texttt{wire b = 1;} \\
        \texttt{wire c[2] = 01b;} \\
        \texttt{//\{a,b,c,01b\} results in 010101b} \\
        \texttt{wire a1 = 1;} \\
        \texttt{wire b1[4];} \\
        \texttt{b1 = \{4\{a1\}\}; \slash\slash results in 1111b, which is equivalent to \{a1,a1,a1,a1\}} \\
        
        \subsubsection{Bitwise}
        Bitwise operators represent the primitive AND, OR, and NOT gates. All other logical are a combination 
        of these operations. Every bitwise operation with the exception of the NOT gate is a binary operation 
        in the “operand operation operand” style with both operands being the same size which is also the 
        size of the return value. A NOT operation will return the same number of bits as its single operand. \\\\
        Primitive bitwise operators ordered by precedence. \\\\
        \texttt{wire a = 01b;} \\
        \texttt{wire b = 11b;} \\\\
        Bitwise Operations: \\
        
        \begin{center}
        \begin{tabular}{|l|>{\texttt\bgroup}l<{\egroup}|>{\texttt\bgroup}l<{\egroup}|}
        \hline
        Operator&Example&Result\\ \hline
        NOT		&	!a		&	10b\\ \hline
        AND		&	a\&b		&	01b\\ \hline
        OR		&	a$\mid$b	&	11b\\ \hline
        \end{tabular}
        \end{center} 
        Full range of bitwise operators: \\
         
        \begin{center} 
        \begin{tabular}{|l|>{\texttt\bgroup}l<{\egroup}|>{\texttt\bgroup}l<{\egroup}|>{\texttt\bgroup}l<{\egroup}|}
        \hline
        Operator&Example&Equivalency&Results\\ \hline
        NAND	&	a!\&b				&		!a \& !b				&	00b		\\ \hline
        NOR		&	a!$\mid$b			&		!a $\mid$ !b			&	10b		\\ \hline
        XOR		&	a\textasciicircum b	&	(!a \& b) $\mid$ (a \& !b)		&			\\ \hline
        XNOR	&	a!\textasciicircum b 	&	(!a $\mid$ b) \& (a $\mid$ !b) 	& 			\\ \hline
        \end{tabular}
        \end{center} 
        
        \subsubsection{Parenthesis}
        Parenthesis has the highest precedence. \\
        \texttt{1$\mid$(1\&0) \slash\slash1} \\
        \texttt{(1$\mid$1)\&0 \slash\slash 0} \\
        
        \subsubsection{Reduction}
        Reduction operators take only a single operand on their right hand side (a bus) and result in a
        single bit result. \\\\
        \texttt{wire a[3] = 010b;} \\
        
        \begin{center} 
        \begin{tabular}{|l|>{\texttt\bgroup}l<{\egroup}|>{\texttt\bgroup}l<{\egroup}|>{\texttt\bgroup}l<{\egroup}|}
        \hline
        Operator&Example&Equivalency&Result\\ \hline
        AND		&	\&a       	&	example[0] \& example[1] \& example[2]     &	-		\\ \hline
        NAND		&	!\&a      	&	!example[0] \& !example[1] \& !example[2]  &	0		\\ \hline
        OR		&	|a       	&	example[0] | example[1] | example[2]       &	1		\\ \hline
        NOR		&	!|a      	&	!example[0] | !example[1] | !example[2]    & 	1		\\ \hline
        XOR		&	\textasciicircum example 	&	example[0] \textasciicircum example[1] \textasciicircum example[2]       &	1		\\ \hline
        XNOR		&	!\textasciicircum example	&	!example[0] \textasciicircum !example[1] \textasciicircum !example[2]    & 	1		\\ \hline
        \end{tabular}
        \end{center}
        
        \subsubsection{Arithmetic}
        Arithmetic operators are shorthand for common equivalent but complex operations. They operate 
        on two bits or buses which do not have to be the same size. They will return a bit or bus. All 
        operations are done in two's complement.\\
        In general, the bus that receives the result must contain enough bits to hold all bits 
        in the result, or the result of the arithmetic operation may be undefined.\\\\
        \texttt{wire e0[3] = 011b //equivalent to 3d and can be expanded to 0011b}\\
        \texttt{wire e1[3] = 111b //equivalent to -1d and can be expanded to 1111b}\\
        \texttt{wire e3[3];}\\
        \texttt{wire e4[4];}\\
        
        \begin{center} 
        \begin{tabular}{|l|>{\texttt\bgroup}l<{\egroup}|p{2in}|>{\texttt\bgroup}l<{\egroup}|}
        \hline
        Operator&Example&Notes&Result\\ \hline
        Plus          	&	e3=e0+e1; e4=e0+e1	&	If there is overflow and the bus holding the result 
        has insufficient bits, the result of the operation may be incorrect. If the resultant has n bits and the
        addends fewer, the addends will both be extended to n bits before addition occurs.     	&
        e3=010b; e4=0010b \\ \hline
        Minus         	&	e3=e0-e1; e4=e0-e1	 & Equivalent to \texttt{e0+{!e1[n],e1[n-1:0]}}.        
                           &	e3 = 100b; e4 = 0100b \\ \hline
        Multiplication	&	e0*e1              &	Returns a bus that is n+m-1 long                                      &
        	10101b                \\ \hline
        Division      	&	e0/e1               &	Not implemented                                                         & 	                      
        \\\hline
        Modulus       	&	e0\%e1             	&	Returns a bus that is n long where n is the size 
        of the first operand   &	0b11                  \\ \hline
        \end{tabular}
        \end{center}
        
        \subsubsection{Shifting}
        Shifting operations will literally shift the entire bus to the left or right and will discard the 
        bits shifted off the end.\\\\
        \texttt{e0 = 0111b; //7d}\\
        \texttt{e1 = 1111b; //-1d}\\
        
        \begin{center} 
        \begin{tabular}{|l|>{\texttt\bgroup}l<{\egroup}|p{2in}|>
        {\texttt\bgroup}l<{\egroup}|}
        \hline
        Operator&Example&Note&Result\\ \hline
        Left-shift &	e0<<2; e1<<2	&	Left shift will always fill with zeros     	&	
        1100b //-4d; 1100b	  \\ \hline
        Right-shift &	e0>>2; e1>>2	&	Right shift will always fill with the most significant bit 
        to preserve sign &	0001b //1d; 1111b //-1d \\ \hline
        \end{tabular}
        \end{center}
        
        \subsubsection{Sign Extension}
        The sign extension operator sign-extends the right operand to the number of bits specified by the 
        left hand side operand in decimal. The left operand must be determinable at compile time.
        Attempts to specify a number of bits that is smaller than the number of bits in the right operand is a 
        syntax error. The operation is done in two�s complement.\\\\
        \texttt{e0 = 0111b; //7d}\\
        \texttt{e1 = 1111b; //7d}\\
        
        \begin{center} 
        \begin{tabular}{|l|>{\texttt\bgroup}l<{\egroup}|p{2in}|>{\texttt\bgroup}l<{\egroup}|}
        \hline
        Operator&Example&Note&Result\\ \hline
        Sign Extension &	8'e0; \ 8'e1	&	Left hand side is always a decimal number.     	&	
        0000011b; \ 11111111b \\ \hline
        \end{tabular}
        \end{center}
        
        \subsubsection{Comparison}
        Comparison operators will compare the values of two buses which do not need to be equally sized 
        and will return a one bit true or false result.\\\\        
        \texttt{e0 = 0111b; //7d}\\
        \texttt{e1 = 0111b; //7d}\\
        \texttt{e2 = 1111b; //-1d}\\
        
        
        \begin{center} 
        \begin{tabular}{|l|>{\texttt\bgroup}l<{\egroup}|>{\texttt\bgroup}l<{\egroup}|}
        \hline
        Operator&Example&Result\\ \hline
        Less than				&	e0 < e1 		&	0 \\ \hline
        Less than or equal to	&	e0 <= e1		& 	1 \\ \hline
        Greater than				&	e0 > e2		& 	1 \\ \hline
        Greater than or equal to &	e >= e2		& 	1 \\ \hline
        Equal to					& e0 == e1		& 	1 \\ \hline
        Not equal to				& e0 != e1		&	0 \\ \hline
        
        \end{tabular}
        \end{center}
        
        
        \subsubsection{Precedence and associativity}
        
        \begin{center} 
        \begin{tabular}{|>{\texttt\bgroup}l<{\egroup}|l|}
        \hline
        Operators, in order of decreasing precedence &Associativity\\ \hline
        ! \ + \ - \ \& \ !\&	 \ \^ \ !\^ 	\ $\mid$ \ !$\mid$ (unary) \ ' (sign extension) &	 right to left \\ \hline
        * \% /	& 	left to right \\ \hline
        	+ - (binary)		& 	left to right \\ \hline
        << >> & 	left to right \\ \hline
        < <= > >=					& 	left to right \\ \hline
        == !=				&	left to right \\ \hline
        \& \ !\& (binary)				&	left to right \\ \hline
        \^ \ !\^ \ (binary)		&	left to right \\ \hline
        $\mid$ !$\mid$ (binary)				&	left to right \\ \hline
        =				&	right to left \\ \hline
        
        \end{tabular}
        \end{center}
        
    \subsection{Declarations}
    There are three types of declarations in Verishort: wires, for passing values between
    modules; registers, for storing values between clock cycles; and modules, blocks of
    Verishort code offering specific functionality.\\
    
        \subsubsection{Identifiers}
        Identifiers are user-friendly names, much like variable names in most programming
        languages.  They must start with an upper- or lower-case letter followed by any 
        sequence of letters, numbers, and underscores.  No other characters are allowed in
        identifier names.\\
        
        \subsubsection{Wire Declarations}
        Wires can be single or multi-bit (buses/bundles).  They are declared using the keyword
        \texttt{wire} followed by a space, then an identifier, then an optional number inside
        square brackets.  The number inside the brackets is the number of bits to be used for the 
        bus, using 0-indexing.  The MSB is at the highest bit value, i.e., at index 15 for a 16-bit
        wire.  If the brackets and enclosed number are omitted, the wire is assumed to hold a
        single bit.\\\\
        \texttt{wire w1; \tab // single bit}\\
        \texttt{wire w2[5]; // five bits with MSB at index 4, LSB at index 0};\\\\
        In addition, the declared wires can take values immediately during the declaration. 
        Following the identifier (or optional bracketed expression) another space, an equals 
        sign, a space, a binary, decimal, or logical value can be used:\\\\
        \texttt{wire w3 = !w2[0];}\\
        \texttt{wire w4[3] = 4d;}\\\\
        The shorthand for digit repetition can also be used:\\
        \texttt{wire w5[16] = {1,14{0},1};}\\\\
        Subsets of buses can be assigned:\\
        \texttt{wire w6[8] = w5[0:7];}\\\\
        Equivalently:\\
        \texttt{for(i = 0; i < 8; i = i + 1) \{\\
        \tab w6[i] = w5[i];\\
        \}}\\
        
        \subsubsection{Register Declarations}
        Registers are declared exactly like wires but with the \texttt{register} keyword.
        The main difference between registers and wires are that registers can be reassigned 
        at will at any point in your code.  They can be used to store values between clock 
        cycles, for example.\\
        
        \subsubsection{Module Declarations}
        Modules are analogous to classes in Java.  Modules allow code to be grouped to offer
        specific functionality.\\
        They are declared with the \texttt{module} keyword, followed by an identifier, followed by a
        parenthesized expression.  The parenthesized expression contains a comma-separated list of
        input ports preceded by the \texttt{input} keyword, followed by a semi-colon, followed by a
        comma-separated list of output ports preceded by the \texttt{output} keyword.  The body of
        the module (discussed below) followed inside braces:\\
        
        \texttt{module m1(input in1, in2, in3[3]; output out[5]) \{ ... \} }\\\\
        In addition to the user-specified inputs and outputs (see the \emph{Assignment} section for
        information on binding ports), each module has an implicit reset input (keyword \texttt{reset})
        and clock input (keywords \texttt{posedge} and \texttt{negedge}).\\\\
        
        Modules can be declared in any order but all code must reside in one file.  Modules cannot be 
        nested, i.e., one module cannot be declared inside of another module.  However, a module can be 
        instantiated (\emph{Instantiating A Module} below) inside of another module as long as there are 
        no infinitely recursive references.
        
        
        \subsubsection{Building A Module}
        A module is a series of declarations, assignments, logic, and conditional statements. 
        All declarations in a module must appear at the beginning of the module, before any other statements.\\
        Here is a brief example:\\\\
        \texttt{
        module m2(input enable, in[4]; output out[4]) \{ \\
        \tab register r[4] = 0; \\
        \tab if(posedge) \{ \\
        \tab\tab if(reset) \{ \\
        \tab\tab\tab r = 0000b; // equivalent to r = 0d; \\
        \tab\tab \} \\
        \tab \} \\
        \tab else if(enable == 1b) \{ \\
        \tab\tab r = in; \\
        \tab\tab out = r; \\
        \tab\} \\
        \}}
        
        This is a simple four-bit latch. On every positive clock edge, if the enable bit is set to 1, 
        the output will be set to the input. If the enable bit is set to 0, the output will remain 
        unchanged. However, if the module's reset bit is 1 when the clock edge is detected, the 
        register values and output will be reset to \texttt{0d}.\\\\
        This module is reusable. Here is an example of instantiating the module from within another module: \\\\
        \texttt{
        module m3(input check1, check2, in[4]; output out[4]) \{ \\
        \tab wire enabler, resetter; \\
        \tab m2(enable = enabler, in = in; out = out; reset = resetter); \\\\
        \tab if(posedge) \{ \\
        \tab\tab enabler = check1; \\
        \tab\tab resetter = check1 \& check2; \\
        \tab \}  \\
        \}}
        
        The enable bit of module \texttt{m2} is 1 if input \texttt{check1} in module \texttt{m3} is 1. If both 
        \texttt{check1} and \texttt{check2} are 1, then the module m2 is reset. check2 on its own does nothing.\\\\
        Notice that though reset is not listed as an input of module  \texttt{m2}, it is implicit; it can be 
        referred to by keyword \texttt{reset}.	\\\\
        In addition to binding wires to outputs, subsets of outputs can be returned as in traditional 
        languages.  To indicate that a module returns \emph{n} bits, put the number of bits to be returned 
        (as in a \texttt{wire} declaration) in square brackets after the identifier, including for a single 
        bit returned, unlike wires.\\\\  
        If you wish to return all or a subset of outputs (replacing or in addition to the normal outputs 
        via binding), list them after the \texttt{return} keyword anywhere in a block.  A single block may not 
        have multiple \texttt{return} keywords. \\\\
        \texttt{
        module m10[5] (input in[5]; output out[3]) \{ \\
        \tab ... \\
        \tab out = ... ; \\
        \tab return \{out,in[1],in[2]\}; \\
        \}}\\\\
        These can then be used as follows:
        \texttt{wire w11[5] = m10(...);}
        
        \subsubsection{Statements}
        A semicolon is necessary after a statement in Verishort.  Because whitespace has no effect, it is 
        necessary to have semicolons to signal the end of a statement.\\\\
        Examples from previous sections:\\
        \texttt{wire w3 = !w2[0];}\\
        \texttt{wire c[2] = 01b;} \\\\
        For some statements, such as connectivity, a statement is inherently acknowledged through the brackets.\\\\
        Example from previous sections:\\
        \texttt{wire a = 1;\\
        wire b[4];\\
        wire c[16] = \{0\{b\},0,0,10b,4\{0,a\},b\} \tab // results in 0010010101011111b \\
        b = \{4\{a\}\};	\tab\tab\tab // results in 1111b}\\\\
        
        \subsubsection{Conditional Statements}
        Conditional statements work just as they do in the C programming languages with \texttt{if} and \texttt{else}. 
        Following an \texttt{if}, an expression is placed within parenthesis.  This expression must evaluate to a bit 
        value.  An expression returning 1 evaluates as the equivalent of �true� and 0 as �false�.  An \texttt{else} 
        block attaches itself to the closest \texttt{else}-less \texttt{if} block.\\\\
        \texttt{wire gate = 1; \\
        wire b[2]; \\
        if (gate \& 1b > 0b) \{ \\
        \tab b = 10b; \\
        \} \\
        else \{ \\
        \tab b = 01b; \\
        \}}\\\\
        The power of conditional statements come in their ability to use the clock.  For example, an incrementer:\\\\
        \texttt{
        register a[8] = 0; \\
        if (posedge) \{ \\
        \tab a = a + 1b;\\
        \}}\\\\
        Asynchronous and synchronous logic can also be combined together seamlessly:\\\\
        \texttt{
        wire gate = 1; \\
        register a[8] = 0; \\
        if (posedge \& gate) \{ \\
        \tab a = a + 1b; \\
        \}}\\\\
        
        \subsubsection{Case\slash Switch Structures}
        Case structures use the \texttt{case} keyword and work similarly to the switch statement in C.  The main 
        difference is that the case structure in Verishort does not provide fall-through or default behavior.\\\\
        This is especially useful when the user wants to test for conditions on certain bits but doesn�t care 
        about the value of other bits.  Those bits can be replaced with \texttt{x} instead of 1 or 0.\\\\
        Inside of a case block, the condition is followed by a colon, then followed by the resulting statement, 
        followed by a semicolon.\\\\
        If-else statements and for-loops cannot appear inside case structures: \\\\
        \texttt{
        wire w[3] = ... \tab // also assume a 3-bit output �out� \\
        case(w) \{ \\
        \tab 1x1b : out = 111b; out2= 110b; \\
        \tab x00b : out = 000b; \\
        \}} \\\\
        This code has the effect of creating 16 \texttt{example\_modules} which are all wired.
    \subsection{Scope}
    VeriShort tends to be a linear language, with very little dependence on lexical scope and linkage. 
    That being said, VeriShort still limits scope for certain conditions.  Importantly, data declared 
    within one module is not available outside of that module.  Similarly, data declared within blocks 
    (between the brackets in \texttt{if-else} blocks and \texttt{for} loops) is not available
    outside of those blocks.
    
    \subsection{Preprocessor}
    Like the \#define directive in C, the \texttt{parameter} keyword can be used in VeriShort to replace 
    numbers before compile time.  For example, in the following code:\\\\
    \texttt{
    parameter const = 1010b; \\
    wire c[4] = const;}\\\\
    The const identifier is literally replaced with the number 1010b before compile time.  As with any assignment, 
    notice that \texttt{c} contains the correct number of bits to hold the constant.\\\\
    To reduce the repetitiveness of Verilog, we automatically assume clock and reset ports in all modules that 
    use their functionality.  So, a clock input port is included in any module that uses the \texttt{posedge} 
    or \texttt{negedge} keywords.  Similarly, a reset input port is included in any module that uses registers 
    to the effect that asserting a reset will set all registers back to 0.  These ports can be overloaded by 
    simple specifying a different value for them.  This is useful for gating the clock like in the following 
    example:\\\\
    \texttt{
    module gater(input gate, i; output o) \{ \\
    \tab add\_on\_clk(in=i; out=o); \\
    \tab clock = clock \& gate; \\
    \}}\\\\
    Assuming that the \texttt{add\_on\_clk} module actually uses a clock, this also has the effect of changing the clock 
    definition in the \texttt{add\_on\_clock} and any other module instantiated within the gater module.\\\\
    
    \subsection{Standard Library Modules}
    Because of the repetitiveness of many aspects of hardware design, the Verishort language includes standard modules 
    of many commonly used electronic components in the hope of reducing some tedium. The definitions given below are 
    pseudo-module declarations. The appropriate value of n will be selected at compile time.  Note that, like any 
    module port, clk and reset do not need to be specified.  By Verishort convention, a \texttt{clk} is explicitly assumed 
    in all modules that use the \texttt{posedge/negedge} keywords.  A reset is explicity assumed in all modules that use the 
    reset keyword.
    
        \subsubsection{Latches}
        \texttt{module SRL[n] (input S[n], reset[n], E[n]; output Q[n], QNOT[n]) \\
        module JKL[n] (input J[n], K[n], E[n]; output Q[n], QNOT[n]) \\
        module DL[n] (input D[n], E[n]; output Q[n], QNOT[n]) \\
        module TL[n] (input T[n], E[n]; output Q[n], QNOT[n]}
        
        \subsubsection{Flipflops}
        \texttt{module DFF[n] (input D[n], S[n], reset[n], clk; output Q[n], QNOT[n]) \\
        module TFF[n] (input T[n], clk; output Q[n], QNOT[n]) \\
        module JKFF[n] (input J[n], K[n], clk; output Q[n], QNOT[n])}
        
        \subsubsection{Multiplexers}
        \texttt{module MUX[1] (input IN[n],SEL[log2 n]; output OUT) \\
        module DEMUX[n] (input IN[n], SEL[log2 n]; output OUT[n])}
        
        \subsubsection{Decoders}
        \texttt{module DECODE[n] (input IN[log2 n]; output OUT[n]) \\
        module ENCODE[log2 n] (input IN[n]; output OUT[log2 n]) //Priority Encoder}

\section{Project Plan}
    \subsection{Team Responsibilities}
        \subsubsection{Round-Robin Dictatorship}
        At some point in this project, each of our members has been our dictator. Leadership switched hands especially when
        moving between important phases in the project. For example, Elba was dictator for the failed nation of Natural. Anish
        usurped the leadership position soon after Verishort was established. Next was Scott, when regular meetings started
        getting iffy during midterms. Soon after, our final and present dictator took hold, our Dear Leader, Tim Song.\\\\
        We mostly worked together during the large blocks of time which we held our meetings. All sections were written with
        all members present to catch mistakes or give suggestions. 
    \subsection{Project Timeline}
	Week of September 13, 2010 - Our team got together, and brainstormed.
	Week of September 20, 2010 - Natural, our first attempt at a project, was born.
	Week of September 27, 2010 - A written proposal for Natural was submitted.
	Week of October 4, 2010 - No-op.
	Week of October 11, 2010 - Met with Professor Edwards and got recommended to establish a new project.\\
	Began looking into Verilog a possible project replacement. 
	Week of October 18, 2010 - Verishort, our eventual final project was born.
	Week of October 25, 2010 - Frantic writing of Language Reference Manual Begins.
	Week of November 1, 2010 - Final writing of LRM finished and submitted. 
	Week of November 8, 2010 - Initial coding of grammar written. 
	Week of November 15, 2010 - 
	Week of November 22, 2010 - 
	Week of November 29, 2010 -
	Week of December 6, 2010 -
	Week of December 13, 2010 - 
	Week of December 20, 2010 -  

    
    \subsection{Software Development Environment}
	GEDIT!
    \subsection{Project Log}
    
\section{Architectural Design}
    \subsection{Architecture}
    \subsection{the Runtime Environment}
    \subsection{Error Recovery}

\section{Testing Plan}
    \subsection{Goals}
    \subsection{Hypothesis}
    \subsection{Methods}
        \subsubsection{Phase I}
        \subsubsection{Phase II}
        \subsubsection{Phase III}
    \subsection{Tools}
    \subsection{Implementation}
        \subsubsection{Phase I}
        \subsubsection{Phase II}
        \subsubsection{Phase III}
        
\end{document}
